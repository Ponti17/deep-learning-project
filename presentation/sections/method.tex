\documentclass[../main.tex]{subfiles}

\graphicspath{{\subfix{../imgs/}}}

\begin{document}

\section{Method}
\begin{frame}[t]
    \frametitle{Method}

    \begin{block}{Augmentation}
        \begin{itemize}
            \item Data augmentation is a technique to increase the size of 
            the training set by applying transformations to the original images
            \item The transformations are applied randomly to the images
            \item Examples of transformations:
            \begin{itemize}
                \item Random rotation
                \item Random horizontal flip
                \item Random vertical flip
                \item Random zoom
            \end{itemize}
        \end{itemize}
    \end{block}

    \begin{block}{HoVer-Network}
    We are going to use the HoVer-Net architecture to segment the nuclei in the images.
    \end{block}
\end{frame}

\begin{frame}[t]
    \frametitle{Method}
    \begin{block}{LUMI}
        We are going to use LUMI to train the model. 
        Futrhermore, we believe that LUMI will help us 
        to improve the selection of the hyperparameters and 
        the augmentation techniques.
    \end{block}
    

\end{frame}

\end{document}